\documentclass[11pt]{article}
\usepackage{amssymb, amsmath}
\usepackage{graphicx}

\begin{document}
\textbf{{\huge Nguyen Huu Thanh}}


\textit{\textbf{Assignment 4 - Matrix}}
\vspace{5mm}

Denote matrix $M = \sigma A$. Then, \\
\begin{equation*} 
\begin{split}
X_1 &= IX_0 + MX_0
 = (I + M)X_0\\ 
X_2 &= X_1 + M^2X_0 
= (I + M)X_0 + M^2X_0 
= (I + M + M^2)X_0\\
...\\
X_t &= (I + M + M^2 + ... + M^t)X_0 \\
(M-I)X_t &= (M-I)(I + M + M^2 + ... + M^t)X_0 \\
(M-I)X_t &=(M - I + M^2 - M + M^3 - M^2 + ... + M^{t+1} - M^t)X_0\\
(M-I)X_t &=(M^{t+1} - I) X_0\\
X_t &= (M-I)^{-1}(M^{t+1} - I)X_0
\end{split}
\end{equation*}

By decomposing matrix $M$, we get$M = PDP^{-1}$, where $D$ is a diagonal matrix formed from the eigenvalues of $M$, and the columns
of $P$ are the corresponding eigenvectors of $M$. \\


Because $P^{-1}P = I$ then $M^{t+1}=(PDP^{-1})^{t+1} = PD^{t+1}P^{-1}$.\\

Therefore, $X_t = (M-I)^{-1}(PD^{t+1}P^{-1} - I)X_0$


In this problem, \\
\[ D = 
\begin{bmatrix}
w_1 &0 &0 &0 &0\\
0 &w_2 &0 &0 &0\\
0 &0 &w_3 &0 &0\\
0 &0 &0 &w_4 &0\\
0 &0 &0 &0 &w_5
\end{bmatrix}
\]

Then, \\
\[ D^{t+1} = 
\begin{bmatrix}
w_1^{t+1} &0 &0 &0 &0\\
0 &w_2^{t+1} &0 &0 &0\\
0 &0 &w_3^{t+1} &0 &0\\
0 &0 &0 &w_4^{t+1} &0\\
0 &0 &0 &0 &w_5^{t+1}
\end{bmatrix}
\]
For $w_i$, we have: \\

$w_i^{\infty} = \begin{cases}  0, &\mbox{if }  -1 < w_i < 1\\ 1 \mbox{ or } -1, & \mbox{if } w_i = -1  \mbox{ or } 1
\\ \infty, &\mbox{if } w_i < -1 \mbox{ or } w_i > 1
 \end{cases}$\\   
We consider two cases:
\begin{itemize}
\item One of $w_i \rightarrow \infty$ then $X_{\infty}$ is \textbf{divergent}. \\This happens when $1 > \sigma \geq$ \textbf{\textit{threshold}  $=0.47$}. The threshold is calculated by generating $w_i$ on several values of $\sigma$ in range (0,1).
\item All of $w_i \rightarrow 0$ then $D^{\infty}$ is zero matrix. Then, 
\begin{equation*} 
\begin{split}
X_{\infty} &= (M-I)^{-1}(PD^{\infty}P^{-1} - I)X_0 \\
X_{\infty} &= -(M-I)^{-1}X_0
\end{split}
\end{equation*}
This happens when $0 < \sigma <$ \textit{threshold}.

\end{itemize}
The below result is obtained in case $\sigma = 0.33$.\\
\[X_{\infty} = 
\begin{bmatrix}
1.6322411 \\
0.68862128\\
0.53863956\\
0.45449004\\
0.68862128
\end{bmatrix}
\]



\end{document}